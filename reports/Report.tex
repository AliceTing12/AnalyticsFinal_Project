% Options for packages loaded elsewhere
\PassOptionsToPackage{unicode}{hyperref}
\PassOptionsToPackage{hyphens}{url}
\documentclass[
]{article}
\usepackage{xcolor}
\usepackage[margin=1in]{geometry}
\usepackage{amsmath,amssymb}
\setcounter{secnumdepth}{-\maxdimen} % remove section numbering
\usepackage{iftex}
\ifPDFTeX
  \usepackage[T1]{fontenc}
  \usepackage[utf8]{inputenc}
  \usepackage{textcomp} % provide euro and other symbols
\else % if luatex or xetex
  \usepackage{unicode-math} % this also loads fontspec
  \defaultfontfeatures{Scale=MatchLowercase}
  \defaultfontfeatures[\rmfamily]{Ligatures=TeX,Scale=1}
\fi
\usepackage{lmodern}
\ifPDFTeX\else
  % xetex/luatex font selection
\fi
% Use upquote if available, for straight quotes in verbatim environments
\IfFileExists{upquote.sty}{\usepackage{upquote}}{}
\IfFileExists{microtype.sty}{% use microtype if available
  \usepackage[]{microtype}
  \UseMicrotypeSet[protrusion]{basicmath} % disable protrusion for tt fonts
}{}
\makeatletter
\@ifundefined{KOMAClassName}{% if non-KOMA class
  \IfFileExists{parskip.sty}{%
    \usepackage{parskip}
  }{% else
    \setlength{\parindent}{0pt}
    \setlength{\parskip}{6pt plus 2pt minus 1pt}}
}{% if KOMA class
  \KOMAoptions{parskip=half}}
\makeatother
\usepackage{graphicx}
\makeatletter
\newsavebox\pandoc@box
\newcommand*\pandocbounded[1]{% scales image to fit in text height/width
  \sbox\pandoc@box{#1}%
  \Gscale@div\@tempa{\textheight}{\dimexpr\ht\pandoc@box+\dp\pandoc@box\relax}%
  \Gscale@div\@tempb{\linewidth}{\wd\pandoc@box}%
  \ifdim\@tempb\p@<\@tempa\p@\let\@tempa\@tempb\fi% select the smaller of both
  \ifdim\@tempa\p@<\p@\scalebox{\@tempa}{\usebox\pandoc@box}%
  \else\usebox{\pandoc@box}%
  \fi%
}
% Set default figure placement to htbp
\def\fps@figure{htbp}
\makeatother
\setlength{\emergencystretch}{3em} % prevent overfull lines
\providecommand{\tightlist}{%
  \setlength{\itemsep}{0pt}\setlength{\parskip}{0pt}}
\usepackage{bookmark}
\IfFileExists{xurl.sty}{\usepackage{xurl}}{} % add URL line breaks if available
\urlstyle{same}
\hypersetup{
  pdftitle={Learning Analytics for a FutureLearn MOOC},
  pdfauthor={ALICE TING},
  hidelinks,
  pdfcreator={LaTeX via pandoc}}

\title{Learning Analytics for a FutureLearn MOOC}
\author{ALICE TING}
\date{2026-01-05}

\begin{document}
\maketitle

The full reproducible project, code, and data processing scripts are
available at:
\url{https://github.com/AliceTing12/AnalyticsFinal_Project} \#
Introduction Learning analytics aims to use data generated by learners
to better understand patterns of engagement and support learning
outcomes. In large-scale online courses, traditional measures such as
attendance are not available, making platform interaction data a key
source of insight.This project analyses data from multiple runs of a
FutureLearn MOOC on cyber security. The analysis focuses on the
relationship between learner engagement and course completion, and
follows the CRISP-DM framework through two consecutive analytical
cycles.

\section{CRISP-DM Cycle 1}\label{crisp-dm-cycle-1}

\subsection{Business Understanding}\label{business-understanding}

The primary stakeholders in this analysis are FutureLearn course
designers, instructors, and learning analytics teams. These stakeholders
are interested in understanding how learners engage with course content
and which behaviours are associated with successful course completion.
In large-scale online courses, high enrolment but low completion is
common, making it difficult to identify learners who may be at risk of
disengagement.

The aim of this investigation is to examine whether patterns of learner
engagement, measured through platform activity data, are associated with
course completion. This analysis seeks to support stakeholders in making
evidence-based decisions about course design and learner support
strategies. Success for this project is defined by identifying clear and
interpretable relationships between learner activity and course
completion, and by demonstrating whether these relationships remain
consistent across two CRISP-DM analytical cycles.

\subsection{Data Understanding}\label{data-understanding}

The dataset consists of raw CSV files exported from the FutureLearn
platform for seven runs of a cyber security MOOC. Each run contains
multiple data tables capturing enrolment information, learner activity,
and survey responses. An initial inventory of the raw files showed that
not all table types were consistently available across runs.To ensure
comparability, this analysis focused on enrolment and step activity
data, which were present for all course runs and directly related to
learner engagement.

After data preparation, the core analytical dataset contained 37,296
learner records and 17 variables. Each row represents a learner
enrolment in a specific course run. The dataset includes learner
identifiers, enrolment and participation timestamps, demographic
descriptors, and a derived engagement measure.

Key variables used in this analysis include:

\begin{itemize}
\tightlist
\item
  \texttt{learner\_id} -- unique learner identifier\\
\item
  \texttt{run} -- course run number\\
\item
  \texttt{fully\_participated\_at} -- used to derive course completion\\
\item
  \texttt{total\_activity} -- total recorded activity events per
  learner\\
\item
  \texttt{completed} -- logical variable indicating course completion
\end{itemize}

The dataset contains a mixture of character variables (e.g., timestamps
and demographics), integer variables (activity counts), and logical
variables (completion status).

Initial inspection revealed substantial missing values in demographic
fields and a large number of learners with zero recorded activity. These
data quality characteristics informed both the selection of variables
and the refinement of the second analysis cycle.

\subsection{Data Preparation}\label{data-preparation}

To prepare the data for analysis, enrolment records from all course runs
were combined into a single dataset, with each learner linked to a
specific run. Activity log files were then aggregated at the learner
level to produce a simple measure of engagement based on the total
number of recorded activity events. These engagement summaries were
joined to enrolment data using learner identifiers, resulting in a
single learner-level dataset suitable for exploratory analysis.

\subsection{Exploratory Analysis}\label{exploratory-analysis}

\pandocbounded{\includegraphics[keepaspectratio]{Report_files/figure-latex/activity-boxplot-cycle1-1.pdf}}

This figure compares total recorded activity between learners who
completed the course and those who did not. Activity levels are highly
skewed, so a logarithmic scale is used to allow meaningful comparison
across learners with very different engagement levels. Learners who
completed the course generally show substantially higher levels of
activity than those who did not complete. Non-completing learners are
concentrated at very low activity levels, while completing learners tend
to exhibit consistently higher engagement. This suggests a strong
association between sustained engagement and course completion.

\subsection{Evaluation}\label{evaluation}

The results from the first analysis cycle indicate a clear relationship
between learner activity and course completion. However, this analysis
is exploratory and does not establish causality. Completion was
approximated using available participation indicators, which may not
fully capture learning outcomes. In addition, activity measures reflect
platform interaction rather than learning quality.The success criteria
were met, as clear and interpretable differences in engagement between
completing and non-completing learners were observed.

\subsection{Deployment}\label{deployment}

From a practical perspective, these findings suggest that early
engagement data could be used to identify learners at risk of
disengagement. Monitoring activity levels during the early stages of the
course may enable timely interventions, such as reminder emails or
additional support resources. Such approaches could help improve learner
retention in future course runs.

\section{CRISP-DM Cycle 2}\label{crisp-dm-cycle-2}

\subsection{Revised Business
Understanding}\label{revised-business-understanding}

Results from Cycle 1 showed that a large proportion of enrolled learners
recorded little or no activity. This raised the concern that the
observed relationship between engagement and completion might be driven
primarily by inactive enrolments rather than meaningful differences
among participating learners.

The second analysis cycle therefore focuses on learners who engaged with
the course at least once. The refined objective is to examine whether
the relationship between engagement and completion remains evident when
only active learners are considered.

\subsection{Revised Data Preparation}\label{revised-data-preparation}

For the second analysis cycle, learners with zero recorded activity were
excluded from the dataset. This refinement ensured that the analysis
focused on learners who engaged with at least some course content. No
additional data sources were introduced, and the same engagement
measures were retained.

\subsection{Exploratory Analysis
(Refined)}\label{exploratory-analysis-refined}

\pandocbounded{\includegraphics[keepaspectratio]{Report_files/figure-latex/activity-boxplot-cycle2-1.pdf}}

This refined figure shows total recorded activity for learners who
recorded at least one activity event. Compared to the initial analysis,
removing inactive learners reduces extreme skew at the lower end of the
distribution. Despite this refinement, learners who completed the course
continue to show higher levels of activity than those who did not
complete. This indicates that the relationship observed in Cycle 1 is
not solely driven by learners who never engaged with the course.

\subsection{Evaluation}\label{evaluation-1}

The refined analysis supports the findings from Cycle 1 and strengthens
confidence in the observed relationship between engagement and course
completion. By focusing on active learners, the analysis reduces the
influence of non-participating enrolments. However, the results remain
descriptive and do not account for factors such as learner motivation,
prior knowledge, or external constraints. As such, the findings should
be interpreted as indicative rather than predictive.

\section{Discussion and Limitations}\label{discussion-and-limitations}

This analysis examined the relationship between learner engagement and
course completion across multiple runs of a FutureLearn MOOC using two
CRISP-DM cycles. Results from both cycles consistently showed that
learners who completed the course tended to exhibit higher levels of
recorded activity. Refining the analysis in Cycle 2 to focus on active
learners confirmed that this relationship was not driven solely by
inactive enrolments, strengthening confidence in the findings. However,
the analysis relies on platform interaction data, which measures
quantity rather than quality of engagement. In addition, factors such as
learner motivation, prior experience, and external commitments were not
available and may influence both engagement and completion.

\section{Conclusion}\label{conclusion}

This project applied the CRISP-DM framework to explore learner
engagement and course completion in a large-scale online course. Across
two analytical cycles, higher levels of learner activity were
consistently associated with course completion. Refining the analysis to
exclude inactive learners demonstrated that this relationship was robust
and not solely driven by non-participating enrolments. While the
findings are descriptive rather than predictive, they highlight the
potential value of engagement data for identifying learners at risk of
disengagement. Future work could extend this analysis by examining
temporal patterns of activity or incorporating additional learner
characteristics to support more targeted learning analytics
interventions.

\end{document}
